% --- Layout e formattazione generale ---
\usepackage[a4paper, top=1.5cm, bottom=1.5cm, left=2cm, right=2cm]{geometry}
\setlength\parindent{0pt}

% --- Matematica ---
\usepackage{amsthm}
\usepackage{amsmath}
\usepackage{amsfonts}
\usepackage{amssymb}
\usepackage{cancel}
\usepackage{arydshln}
\usepackage{xcolor}
\numberwithin{equation}{section}

% --- Teoremi, definizioni e ambienti ---
\renewcommand*{\proofname}{Dimostrazione.}
\renewcommand\qedsymbol{$\blacksquare$}
\newtheorem*{definition}{\color{red}\textbf{Definizione}}
\newtheorem*{theorem}{\color{green}\textbf{Teorema}}
\newtheorem*{corollary}{\color{violet}\textbf{Corollario}}
\newtheorem*{oss}{Osservazione}
\newenvironment{example}
{\begin{center}
        \begin{tabular}{|p{0.9\textwidth}|}
            \hline \\ 
            \textit{Esempio}: \\\\ 
        }
        {
            \\\\ \hline
        \end{tabular}
    \end{center}
}
\newenvironment{recap}
{
    \begin{center}
        \begin{tabular}{:p{0.9\textwidth}:}
            \hdashline \\ 
            \textit{Recap}: \\\\ [-1.5ex]
        }
        {
            \\\\ [-1.5ex] \hdashline
        \end{tabular}
    \end{center}
}
\newcommand{\alignedintertext}[1]{%
  \noalign{%
    \vskip\belowdisplayshortskip
    \vtop{\hsize=\linewidth#1\par
    \expandafter}%
    \expandafter\prevdepth\the\prevdepth
  }%
}

% --- Grafica e disegni ---
\usepackage{graphicx}
\graphicspath{ {./images/} }
\usepackage{wrapfig}
\usepackage{tikz}
\usepackage{pgfplots}
\pgfplotsset{compat=1.9}

% --- Codifica e listings ---
\usepackage{listings}
\definecolor{codegreen}{rgb}{0,0.6,0}
\definecolor{codegray}{rgb}{0.5,0.5,0.5}
\definecolor{codepurple}{rgb}{0.58,0,0.82}
\definecolor{backcolour}{rgb}{0.95,0.95,0.92}
\lstdefinestyle{mystyle}{
    backgroundcolor=\color{backcolour},   
    commentstyle=\color{codegreen},
    keywordstyle=\color{magenta},
    numberstyle=\tiny\color{codegray},
    stringstyle=\color{codepurple},
    basicstyle=\ttfamily\footnotesize,
    breakatwhitespace=false,         
    breaklines=true,                 
    captionpos=b,                    
    keepspaces=true,                 
    numbers=left,                    
    numbersep=5pt,                  
    showspaces=false,                
    showstringspaces=false,
    showtabs=false,                  
    tabsize=2
}
\lstset{style=mystyle}

% --- Altri comandi ---
% Ruffini
\ExplSyntaxOn
\NewDocumentCommand{\ruffini}{mmmm}
 {% #1 = polynomial, #2 = divisor, #3 = middle row, #4 = result
  \franklin_ruffini:nnnn { #1 } { #2 } { #3 } { #4 }
 }

\seq_new:N \l_franklin_temp_seq
\tl_new:N \l_franklin_scheme_tl
\int_new:N \l_franklin_degree_int

\cs_new_protected:Npn \franklin_ruffini:nnnn #1 #2 #3 #4
 {
  % Start the first row
  \tl_set:Nn \l_franklin_scheme_tl { & }
  % Split the list of coefficients
  \seq_set_split:Nnn \l_franklin_temp_seq { , } { #1 }
  % Remember the number of columns
  \int_set:Nn \l_franklin_degree_int { \seq_count:N \l_franklin_temp_seq }
  % Fill the first row
  \tl_put_right:Nx \l_franklin_scheme_tl
   { \seq_use:Nn \l_franklin_temp_seq { & } }
  % End the first row and leave two empty places in the next
  \tl_put_right:Nn \l_franklin_scheme_tl { \\ #2 & & }
  % Split the list of coefficients and fill the second row
  \seq_set_split:Nnn \l_franklin_temp_seq { , } { #3 }
  \tl_put_right:Nx \l_franklin_scheme_tl
   { \seq_use:Nn \l_franklin_temp_seq { & } }
  % End the second row
  \tl_put_right:Nn \l_franklin_scheme_tl { \\ \hline }
  % Split and fill the third row
  \seq_set_split:Nnn \l_franklin_temp_seq { , } { #4 }
  \tl_put_right:Nx \l_franklin_scheme_tl
   { & \seq_use:Nn \l_franklin_temp_seq { & } }
  % Start the array (with \use:x because the array package
  % doesn't expand the argument)
  \use:x
   {
    \exp_not:n { \begin{array} } { r | *{\int_eval:n { \l_franklin_degree_int - 1 }} { r } | r }
   }
  % Body of the array and finish
  \tl_use:N \l_franklin_scheme_tl
  \end{array}
 }
\ExplSyntaxOff


\documentclass{article}
\usepackage{amsthm}
\usepackage{amsmath}
\usepackage{amsfonts}
\usepackage{amssymb}
\usepackage{xcolor}
\usepackage[a4paper, total={6in, 8in}]{geometry}
\usepackage{graphicx}
\graphicspath{ {./images/} }

\newtheorem*{definition}{\color{red}\textbf{Definizione}}
\newtheorem*{theorem}{\color{green}\textbf{Teorema}}
\newenvironment{example}
{\begin{center}
        \begin{tabular}{|p{0.9\textwidth}|}
            \hline \\ 
            \textit{Esempio}: \\\\ 
        }
        {
            \\\\ \hline
        \end{tabular}
    \end{center}
}
\setlength\parindent{0pt}

\begin{document}
\section{Aritmetica computazionale}
\subsubsection{Rappresentazione dei numeri reali}
I \textbf{numeri finiti} sono utilizzati dai calcolatori per rappresentare i numeri reali poiché
questi ultimi possono avere un numero infinito di cifre, che i calcolatori, avendo una
memoria limitata, non sono in grado di rappresentare. 

\begin{theorem}[Rappresentazione in base]
    Sia $\alpha$ un numero reale non nullo. Possiamo rappresentare tale numero con una base
    $\beta\geq 2$, un numero intero scelto da noi, nel seguente modo:
    \begin{equation}
        \begin{aligned}
            \alpha&=\pm(\alpha_1\beta^{-1}+\alpha_2\beta^{-2}+\ldots)\beta^p \\ 
            \alpha&=\pm(\sum_{i=1}^{\infty}\alpha_i\beta^{-i})\beta^p
        \end{aligned}
    \end{equation}
    I vari termini dell'uguaglianza vengono detti:
    $$\begin{array}{lll}
        \beta & & \text{base} \\ 
        p & & \text{esponente} \\ 
        \alpha_i & & \text{cifre del numero} \\
        \sum_{i=1}^{\infty}\alpha_i\beta^{-i} & & \text{mantissa}
    \end{array}$$
    Ogni cifra $\alpha_i$ è un numero intero che varia tra 0 e $\beta-1$. Ad esempio, se lavoriamo in base
    10, le cifre saranno numeri interi compresi tra 0 e 9.\\ 
    Per garantire l'unicità della rappresentazione, è necessario che $\alpha_1\neq 0$. 
    Se così non fosse, il numero 13 potrebbe essere rappresentato come 13, 013, 0013, eccetera,
    il che va contro l'unicità della rappresentazione.
\end{theorem}
Possiamo scrivere un numero $\alpha\in\mathbb{R}$ con $\alpha\neq 0$ in due modi:
\begin{enumerate}
    \item \textbf{forma mista}.
        $$\alpha=\begin{cases}
            \pm(0.000\alpha_1\alpha_2\ldots)_\beta & p\leq 0\\
            \pm(\alpha_1\alpha_2\ldots)_\beta & p>0
        \end{cases}$$
    \item \textbf{forma scientifica}. L'idea è quella di spostare il punto decimale al primo numero $\neq 0$ e
        poi moltiplicare il tutto per $\beta^p$ per riportare il numero al suo valore originale.
        $$\alpha=\pm0.\alpha_1\alpha_2\ldots\cdot\beta^p$$
        \begin{example}
            \begin{equation*}
               \begin{aligned}
                   \alpha&=(12.37)_{10} & \alpha&=0.12237\cdot 10^2 \\
                   \alpha&=(0.0045)_{10} & \alpha&=0.45\cdot 10^{-2} \\ 
                         & & &=(4\cdot 10^{-1}+5\cdot 10^{-2})\cdot 10^{-2}
               \end{aligned} 
            \end{equation*}
        \end{example}
\end{enumerate}
\begin{definition}[Numeri finiti]
    L'insieme $\mathbb{F}$ dei numeri finiti è definito come l'insieme dei numeri espressi in base $\beta$
    (dove $\beta\geq 2$), utilizzando $t$ cifre (con $t\geq 1$). Poiché anche l'esponente $p$
    potrebbe essere così grande da non poter essere rappresentato, è necessario limitare
    l'intervallo degli esponenti rappresentabili. Qui, $\lambda$ indica il più piccolo esponente che
    può essere rappresentato e $\omega$ il più grande esponente rappresentabile.
    \begin{equation*}
        \begin{aligned}
            \mathbb{F}(\beta,t,\lambda,\omega)&=\{0\}\cup\{\alpha\in\mathbb{R}:\alpha=\pm0.\alpha_1\alpha_2\ldots\alpha_t\cdot\beta^p\} \\
                              &=\{0\}\cup\{\alpha\in\mathbb{R}:\alpha=\pm(\sum_{i=1}^{t}\alpha_i\beta^{-i})\beta^p\}
        \end{aligned}
    \end{equation*}
\end{definition}
\begin{example}
    
\end{example}
\end{document}
